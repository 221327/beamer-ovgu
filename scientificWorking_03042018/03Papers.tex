\section{Choose paper to present and apply to committee}
\begin{frame}
\frametitle{Choose yor paper to present}
\begin{itemize}
\item Go through the committees proposals
\item Ask your teachers, supervisors
\item Go through the paper suggestions on course webpage
\item Do your own search for a good paper
\item When you choose a paper, apply to the committee.
\item The paper should be in scope of the committee by its proposal.
\item When rejected, choose other committee or contact me.
\item \textbf{Student can not apply to present in his/her own committee.}
\item \textbf{To 23.4.2018 each student have to have per-reviewed paper selected to present.}
\end{itemize}
\end{frame}

\begin{frame}
\frametitle{Acceptance of presenters by committees}
\begin{itemize}
\item Committee must accept the same number of presentation as is the number of its members (quota).
\item Committees will evaluate presenters applications in the order as received by corresponding member.
\item If the quota is not exceeded and someone want to present paper suggested by committee, committee must accept the paper.
\item If someone wants to present already taken paper, commitee rejects the request or suggest another paper.
\item Committee evaluate if particular paper is in its scope and accept or reject the presentation or suggest change.
\item Committee reject all further applications if the quota is exceeded and mail it to the teacher.
\item Committee decide if it accepts/rejects presentation in 3 days.
\item Committee assigns two of its members as reviewers for each presentation.
\end{itemize}
\end{frame}
